\documentclass{article}
\usepackage[utf8]{inputenc}

\title{Sony DPT-RP1 Lab Notebook}
\author{Daniel Eaton}
\date{Updated:8/26/2018}

\begin{document}

\maketitle

\section{Introduction}

This is documentation for a small lab notebook management program based on the Sony DPT-RP1 ''Digital Paper'' tablet. The program currently supports the following features:
\begin{itemize}
    \item Combining multiple pdfs into larger pdf files (Journals).
    \item Associating metadata with entries for easy search.
    \item Autodetection of handwritten rectangles and image placement.
    \item Autodetection of pdf additions/deletions from Journals.
    \item Table of contents (with entry titles).
    \item Index section for keywords.
\end{itemize}

\noindent In the future, I wish to add the following features to the program:
\begin{itemize}
    \item Fix note deletion.
    \item Making the user input more error robust.
    \item GUI for notebook management.
\end{itemize}

\section{Installation}
The python file \textbf{labnotebook.py} and the latex file \textbf{template.tex} provide the basis of this program. Installation of the program, simply consists of moving these files to the right locations and setting up a home directory for your lab notebook. For now, the program may be installed by running \textbf{labnotebook\_installer.py}. Note that the installer will query you for a directory in which to place your lab notebook. This \textbf{Lab\_Notebook} directory may be moved at any time to a new filepath.\\

The installer will create a folder called \textbf{Lab\_Notebook} within the path that you specified. Within this folder will be the following files:\\

\textbf{labnotebook.py:} Contains the core of the software. Be careful if you try to modify this file.
\textbf{template.tex:} A latex template that is used by \textbf{labnotebook.py} when generating journal pdfs.
\textbf{labnotebook.bat/sh:} A batch or bash file which can be run activate the library manager.

\section{Usage}

Using this software revolves around manipulating the \textbf{Lab\_Notebook} directory while the library manager (\textbf{labnotebook.py}) is running in the background. To activate the library manager run \textbf{labnotebook.bat/sh} or simply run \"python labnotebook.py\" from the terminal. While the library manager is active you may perform any of the following actions to edit your journals:\\

\textbf{New Journal:} To make a new journal, simply add a new folder within your \textbf{Lab\_Notebook} directory. This will automatically create a new journal whose name will correspond to the directory's name.\\

\textbf{New Imagenote:} To add a new pdf from the DPT-RP1, add the new pdf to the \textbf{New\_Imagenotes} subdirectory within a journal diretory of your choice. You will be queried for image files to insert at all detected rectangles.\\

\textbf{New Note:} To add a new pdf which does not come from the DPT-RP1, add the new pdf to the \textbf{New\_Note} subdirectory within a journal diretory of your choice.\\

\textbf{Delete Note:} To delete a pdf from your journal, find the note in the \textbf{Notes} subdirectory and delete it. You can delete more than one note at a time in this manner.\\

Note: I recommend that you do not make any but the above alterations to the \textbf{Lab\_Notebook} directory or its subdirectories while the library manager is running. Doing so may result in corruption of the library filesystem.

\end{document}